% Options for packages loaded elsewhere
\PassOptionsToPackage{unicode}{hyperref}
\PassOptionsToPackage{hyphens}{url}
%
\documentclass[
]{article}
\usepackage{lmodern}
\usepackage{amssymb,amsmath}
\usepackage{ifxetex,ifluatex}
\ifnum 0\ifxetex 1\fi\ifluatex 1\fi=0 % if pdftex
  \usepackage[T1]{fontenc}
  \usepackage[utf8]{inputenc}
  \usepackage{textcomp} % provide euro and other symbols
\else % if luatex or xetex
  \usepackage{unicode-math}
  \defaultfontfeatures{Scale=MatchLowercase}
  \defaultfontfeatures[\rmfamily]{Ligatures=TeX,Scale=1}
\fi
% Use upquote if available, for straight quotes in verbatim environments
\IfFileExists{upquote.sty}{\usepackage{upquote}}{}
\IfFileExists{microtype.sty}{% use microtype if available
  \usepackage[]{microtype}
  \UseMicrotypeSet[protrusion]{basicmath} % disable protrusion for tt fonts
}{}
\makeatletter
\@ifundefined{KOMAClassName}{% if non-KOMA class
  \IfFileExists{parskip.sty}{%
    \usepackage{parskip}
  }{% else
    \setlength{\parindent}{0pt}
    \setlength{\parskip}{6pt plus 2pt minus 1pt}}
}{% if KOMA class
  \KOMAoptions{parskip=half}}
\makeatother
\usepackage{xcolor}
\IfFileExists{xurl.sty}{\usepackage{xurl}}{} % add URL line breaks if available
\IfFileExists{bookmark.sty}{\usepackage{bookmark}}{\usepackage{hyperref}}
\hypersetup{
  hidelinks,
  pdfcreator={LaTeX via pandoc}}
\urlstyle{same} % disable monospaced font for URLs
\usepackage[margin=1in]{geometry}
\usepackage{graphicx,grffile}
\makeatletter
\def\maxwidth{\ifdim\Gin@nat@width>\linewidth\linewidth\else\Gin@nat@width\fi}
\def\maxheight{\ifdim\Gin@nat@height>\textheight\textheight\else\Gin@nat@height\fi}
\makeatother
% Scale images if necessary, so that they will not overflow the page
% margins by default, and it is still possible to overwrite the defaults
% using explicit options in \includegraphics[width, height, ...]{}
\setkeys{Gin}{width=\maxwidth,height=\maxheight,keepaspectratio}
% Set default figure placement to htbp
\makeatletter
\def\fps@figure{htbp}
\makeatother
\setlength{\emergencystretch}{3em} % prevent overfull lines
\providecommand{\tightlist}{%
  \setlength{\itemsep}{0pt}\setlength{\parskip}{0pt}}
\setcounter{secnumdepth}{-\maxdimen} % remove section numbering

\author{}
\date{\vspace{-2.5em}}

\begin{document}

\begin{center}\rule{0.5\linewidth}{0.5pt}\end{center}

title: ``Tasa de Afiliación Sindical'' author: ``Sebastian Osorio y
Diego Polanco'' date: ``07-05-2021'' fontsize: 11pt header-includes:

\usepackage{floatrow}
  \floatsetup[figure]{capposition=top}
  \usepackage{float}
  \floatplacement{figure}{H}

output: pdf\_document: number\_section: true extra\_dependencies:
{[}``float''{]} -- -

\renewcommand{\figurename}{Figura}
\renewcommand{\tablename}{Tabla}

\hypertarget{introducciuxf3n}{%
\section{Introducción}\label{introducciuxf3n}}

La tasa de afiliación sindical y el tamaño promedio de los sindicatos
son indicadores fundamentales para entender la evolución de la
estructura sindical en un país. A grandes rasgos, proporcionan
información acerca del peso del sindicalismo como proporción del total
de ocupados, y una aproximación gruesa al poder de los sindicatos en
términos de su nivel de concentración.

Debido a su naturaleza, estos indicadores ameritan un análisis
relacional con otras variables tanto cuantitativas como cualitativas
para extraer conclusiones relevantes. Por ejemplo, una tasa de
afiliación elevada no dice nada por sí misma del nivel de legitimidad de
las organizaciones de trabajadores ni de su incidencia en la política
nacional, así como el tamaño promedio no entrega información sobre su
capacidad de presión ante los empleadores ni de la participación de los
trabajadores en sus organizaciones. No obstante, ambos indicadores
constituyen una base ineludible para profundizar en todos esos temas.

Aunque son variables centrales, no existen series sistemáticas de datos
que permitan reconstruir su movimiento de largo plazo, sino que varios
esfuerzos aislados que han abordado periodos específicos, muchas veces
con datos discordantes respecto a ciertos años. El objetivo de la
presente minuta es, por un lado, dar cuenta de las aproximaciones
existentes a la evolución de los indicadores de tasa de afiliación y
tamaño promedio de los sindicatos en Chile, y por otro lado, explicar
metodológicamente una propuesta de empalme que abarca el periodo
comprendido entre los años 1932 y 2010.

Con esto, se espera aportar a la construcción de una base sólida para su
utilización en futuras investigaciones sobre movimiento sindical
nacional.

\hypertarget{la-tasa-de-afiliaciuxf3n-sindical-1932-2010}{%
\section{La Tasa de Afiliación Sindical
(1932-2010)}\label{la-tasa-de-afiliaciuxf3n-sindical-1932-2010}}

La tasa de afiliación consiste en el total de trabajadores
sindicalizados, dividido por la fuerza de trabajo ocupada. Existen por
lo menos 7 aproximaciones a esos datos para Chile que permiten construir
la serie: Dirección del Trabajo (2021), que abarca desde 1990 hasta
2019; Díaz et al.~(2016), desde 1970 hasta 2010, con datos aislados en
1932, 1938, 1942, 1946, 1952, 1958 y 1964; Morris y Oyaneder (1962),
entre 1932 y 1959; Barrera (1980), entre 1956 y 1964; DERTO (1977),
entre 1956 y 1972; y Barret (2001), entre 1981 y 1998.

\begin{figure}
\centering
\includegraphics{afiliacion_sindical_files/figure-latex/figura1-1.pdf}
\caption{Total Afiliados Sindicales (1932-2010). Distintas Fuentes.}
\end{figure}

Como se puede apreciar en la Figura 1, aunque toman periodos diferentes,
las aproximaciones al total de afiliados a organizaciones sindicales
guardan bastante coherencia entre sí, al punto que su superposición es
prácticamente indistinguible gráficamente. Por lo tanto, la propuesta de
empalme es simple y consiste en dar cuenta de los datos más verosímiles
en los años que se superponen.

\begin{figure}
\centering
\includegraphics{afiliacion_sindical_files/figure-latex/figura2-1.pdf}
\caption{Tasa de Afiliación Sindical vs Total de Afiliados (1932-2010).}
\end{figure}

La Figura 2 muestra algunos aspectos relevantes sobre la tasa de
sindicalización:

\begin{itemize}
\item
  Hay dos periodos de aumento explosivo de la sindicalización. El
  primero se registra a partir de 1968 y termina en 1978. El segundo
  comienza en 1984 y termina en 1992.
\item
  Un fenómeno interesante son las cifras posteriores a 1973. Aquí se
  aprecia una muy leve disminución, seguida por un aumento en 1978 y una
  abrupta caída en 1979 y 1980. Se estima que los datos reflejan la
  inercia administrativa de los años posteriores al golpe de Estado en
  cuanto al registro de la afiliación sindical obligatoria (en los
  sindicatos industriales), y una ``limpieza administrativa'' realizada
  en 1978, que reflejó el verdadero estado del sindicalismo en el
  contexto de la represión política.
\item
  Las leves diferencias que se aprecian en el cálculo para algunos años,
  se explican tanto por los cambios estacionales en los meses de
  referencia medidos, como por el tratamiento que se hizo de los
  ``sindicatos en receso'' (P. Morris, 1998).
\item
  Existe una fractura sustancial entre la tasa de afiliación y la
  cantidad absoluta de trabajadores afiliados desde 1980, que se vuelve
  especialmente visible a partir de 1998, en la que el crecimiento de la
  sindicalización no coincide con un aumento en la tasa de afiliación.
  El cambio parece coincidir con el cambio en el Código del Trabajo.
\item
  Es importante notar que estos indicadores refieren a los trabajadores
  del sector privado, que son los únicos que pueden conformar sindicatos
  legalmente.
\end{itemize}

\textbf{Los parrafos siguientes debemos revisarlos en funcion de la
decisión de mostrar otros gráficos o no}

Evidentemente, en la figura 2 la serie de FT siempre se encontrara por
sobre el total de ocupados, dado que el total de la fuerza de trabajo
considera, a los ocupados, desocupados y quienes buscan trabajo por
primera vez por ejemplo. No obstante su diferencia si nos entrega
información sobre la intensidad de uso del factor trabajo, dado que
\(e=ocupados/FT\) donde \(e\) se refiere a la tasa de empleo. Esto es
pertinente para nuestro interés de calcular la tasa de afiliación
sindical con distintos numeradores con tal de comparar el componente
ciclico de la tasa de afiliación versus el crecimiento ``vegetativo'' de
la misma.

La diferencia entre la tasa de afiliación sindical con base en los
ocupados, versus la tasa con base en el total de la fuerza de trabajo da
cuenta de en que medida la tasa de afiliación esta dada por un
componente del ciclo económico y no por un crecimiento propio de la
inclusión de la fuerza de trabajo sea formal o informal, urbana o rural,
al acceso a los derechos colectivos del trabajo.

\hypertarget{tamauxf1o-promedio-de-los-sindicatos}{%
\section{Tamaño Promedio de los
Sindicatos}\label{tamauxf1o-promedio-de-los-sindicatos}}

\begin{figure}
\centering
\includegraphics{afiliacion_sindical_files/figure-latex/figura3-1.pdf}
\caption{Tamaño Promedio de los Sindicatos y Total de Sindicatos
(1932-2010}
\end{figure}

\hypertarget{conclusiones}{%
\section{Conclusiones}\label{conclusiones}}

\end{document}
